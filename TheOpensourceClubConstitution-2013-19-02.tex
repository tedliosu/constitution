%%% Constitution for the Open Source Club at the Ohio State University
%%% distributed under a creative commons 0 (CC-0)
\documentclass{article}
\title{The Open Source Club Constitution}
\author{Original Authors - Isaac Jones, Martin Jansche, and Michael Benedict}
\date{First finalized March 20, 2000\\ Updated October 7, 2017}
\setcounter{secnumdepth}{0}
\usepackage[normalem]{ulem}
\usepackage{graphicx}
\begin{document}
	\maketitle

	\section{Preamble}

	It is important to recognize the danger that bureaucracy can get in the way of doing cool things.  So long as the requirements of the Ohio State University and the realistic needs of the club are met, efforts should be made to minimize bureaucracy.

	As the club has evolved it has become necessary to elect additional officers in order to aid the further growth of the club. While this is not an attempt to create a bureaucracy, it is necessary in order to delegate duties among members in order to help the club run smoothly. At the time of this writing there are at least three elected positions within the club: President (or Benevolent Dictator), Vice President, and Treasurer. These three positions will continue to exist as long as they are required by the Ohio State University.

	\section{Article I - Name, Purpose and Non-Discrimination Policy}

	\subsection{Section 1 - Name}

	This is the Open Source Club at the Ohio State University.

	\subsection{Section 2 - Purpose}

	Our purpose is to write and advocate free software, and to create a community of excellent programmers. We advocate free software through creation of documentation, providing support, and fighting cluelessness of the existence and merits of Open Source Software.

	The club should take no ``official'' stand on political issues in debate among the open source community, should adopt no ``official" club license, should not create strict guidelines for what kind of software should be developed, and should not create strict guidelines for how a software project should be managed (including frequency of updates, release models, code repository management, etc). Doing so would violate the spirit of the Preamble. Individual software projects can be managed however the individuals involved in the project see fit.

	Implicit in this is that the club must allow advocacy of proprietary software written for open source platforms (for instance) or writing of open source software for a proprietary platform (for instance) if some of its members wish to do so. However, programs written by members of the club, under the flag of the club, and distributed by the club should meet some acceptable definition of open source software.

	\subsection{Section 3 - Non-Discrimination Policy}

	In recognition of the importance of welcoming diversity for the sake of creativity, and for the benefit of humanity, this club welcomes all people. This means that our club and its members will not discriminate against any individual(s) for reasons of age, color, disability, gender identity or expression, national origin, race, religion, sex, sexual orientation, or veteran status.

	\section{Article II - Membership: Qualifications and Categories of Membership}

	\subsection{Part 1 - Membership Categories and Selection Processes}
	Ohio State University guidelines demand that voting membership be limited to currently enrolled students.  In order to meet this requirement, membership will be divided into voting and non-voting categories.

	\begin{itemize}
		\item Non-voting membership can be obtained by anyone, student or non-student, simply by attending at least one meeting or being on our mailing list.
		\item Voting membership requires attendance of at least four separate meetings (consecutive or non-consecutive) prior to any given voting event, \textbf{and} verification of being a currently enrolled student.  If attendance is to be taken, it shall be taken via Remind.com using the club account, with the general attendance procedure at meetings outlined in By-Law I below.
	\end{itemize}

Even though by definition a club officer is also a club member, ``member" and ``members" in Article II from this point foward shall mean club members who are \textit{not} club officers.

	\subsection{Part 2 - Membership Privileges}

	Accounts on the club computers can be given out to any member who requests an account, provided Ohio State University's rules are followed.  These accounts may be revoked due to inactivity or failure to follow the rules.  Key card access to the club office can be granted on request given the consent of the president, provided that Ohio State University rules are followed.  Access may be revoked due to inactivity or failure to follow the rules.
	When it is necessary to take a club-wide vote on any issue, only voting members are allowed to do so.

	\subsection{Part 3 - Membership Removal}

	Should cause for membership removal be raised, a member can be removed in one of two ways:
	\begin{enumerate}
		\item A majority vote of club officers.
		\item A two thirds majority vote of voting members, providing there is at least two week's notice, and then another notice a week later, of the said vote.
	\end{enumerate}

	\section{Article III - Officer Positions, Duties, Selection, and Removal}

	\subsection{Part 1 - Officer Positions, Duties, Powers and Limitations.}

	There are three officer positions required by the Ohio State University: The President (or Benevolent Dictator), the Vice President, and the Treasurer.  These three positions will continue to exist as long as they are required by the Ohio State University. Other officer positions may be added and removed as needed, and may be open to election or appointed by the president. The responsibilities of the three aforementioned officers are as follows:\\

	The President shall have the following responsibilities:
	\begin{itemize}
		\item Coordinating with the advisor(s) and the Computer Science and Engineering department's help desk in managing key card access to the office.
		\item Assignment of administrative accounts on the club computers.
		\item Ensuring all software on club computers is compliant with University policy\footnote{At the time of this Constitution update, our advisor Jeremy Morris is the best point of contact with Engineering Technology Services, the overlords of the network which the club computers reside on.}.
		\item \textit{Timely} securing the location of each meeting.
		\item \textit{Timely} announcing the location and topic of each meeting.
		\item Representing the club at other functions and in public.
		\item Appointment and creation of miscellaneous positions.\pagebreak
	\end{itemize}
	
	The Vice President shall have the following responsibilities:
	\begin{itemize}
		\item Keeping the minutes of the meeting.
		\item Keeping track of to-do lists for all officers, including for the Vice President him/herself, for all club-related matters.
		\item Carrying out the duties of the President when the President is unable to do so.\\
	\end{itemize}

	The Treasurer shall the following responsibilities:
	\begin{itemize}
		\item Producing and presenting a quarterly budget.
		\item Purchasing pizza and other consumables for meetings.
		\item Representing our club and recording any important proceedings at all Engineers' Council meetings.
		\item Producing and submiting applications for funding.\\
	\end{itemize}

	All of these responsibilities may be delegated to other club members, though they are ultimately the responsibility of the aforementioned officers.  The President will probably be the hardest working member of the club, and therefore needs the power to make unilateral decisions for the sake of saving time.  None of the officers will necessarily have any control over software projects; by the definition of open source software, code forks can happen when they need to, and therefore no one has ultimate power over a software project.

	\subsection{Part 2 - Officer Selection}

	Once a year, elections will take place for officer position(s).  This election must be announced first at least two weeks prior, with another announcement made one week prior, before the actual voting takes place.  Any voting member (excluding current officers in place up to the beginning of the current election) who meets The Ohio State University's requirements for being a club officer may run for any of the \textit{open} officer positions, which excludes extra positions that are designated as being appointed (see Article III Part 1).  An open officer position is defined as an officer position where there \textit{is} no current officer, or the current officer in place does not wish to continue to be or will unable to be in the said position for the next academic year.  For each respective open officer position, the individual who receives the most votes will attain that position for the next academic year, unless removed.  Ties will be broken by a vote of the current officers in place up to the beginning of the current election.

	\subsection{Part 3 - Officer Removal}

	Leadership is needed in any club and so officers are required to be present at eight out of ten official meetings per quarter, while it is also realized that extenuating circumstances are possible.  In order to eliminate doubts surrounding an officers excuse for missing a meeting, said officer must notify at least one of the other officers that they will not be in attendance at least 24 hours in advance.  If an officer is deemed to be in violation of the aforementioned attendance rules, then the club members will take a vote as to whether or not said officer will be allowed to keep his or her position. For all other cases of misconduct, the officer will face the same removal process as any other member of the club.

	\subsection{Part 4 - Replacing Officers}

	If an officer is either removed or resigns during their term, the remaining officers shall replace the missing officer given that all rules set forth by The Ohio State University are followed.

	\section{Article IV - Advisor: Qualification Criteria}

	The advisor should be a person of technical experience. Preferably someone who has been involved in open source development. The advisor must grok the goals of the club so that he or she does not get in their way. The advisor exists to provide guidance, mentorship, and cluefulness.  The length of term of an advisor is indefinite.  As long as the advisor wishes to continue to be an advisor for the club, and fulfills the duties and requirements enumerated in this section, then the said advisor will remain an advisor for the club.

	University guidelines demand that the advisor (or failing that, the sub-advisor) of student organizations must be ``full-time members of the University faculty or Administrative \& Professional staff.''

	\section{Article V - Meetings of the Organization}

	It should be recognized that those involved in individual software projects will have meetings separate from the meetings of rest of the club. It is encouraged, however, that as many club members as possible be invited to these meetings.

	If none of the officers of the club can be in attendance at a given meeting then said meeting will not take place, as decisions cannot be made without the presiding officers. Members can still hold an independent meeting, but it will not be recognized as an official club meeting.

	\section{Article VI - Method of Amending Constitution: Proposals, notices and voting requirements}

	In the case that someone proposes to amend the constitution, the amendment(s)-in-question must first be written out.  What the admendment(s) is(are) written on/with isn't important; it simply needs to be recorded first.  Next, an announcement must be made first at least two weeks in advance, then a week in advance, in order to notify club members that a club-wide vote will take place to determine whether the said admendment(s) should be ratified. The voting process, if done in person, will involve an officer clearly stating the contents of the admentment(s) before it(they) can be voted on.  Actual voting can be in person or via some electronic medium provided that in either case a reasonable amount of certainty of identity can be secured AND that the voter will have access to a clear description of the admendment.  Amendments and changes should be taken advisedly and considered for a reasonable amount of time before being implemented.

	It is strictly discouraged that the constitution should be amended frequently. If it is to be amended, and by-laws do not exist, this this article (article VI) allows the creation of by-laws in favor of amending the main body of the constitution (this document).

	It is further strictly discouraged that any amendments or by-laws restricting free commerce and creativity be created. New amendments or by-laws should not conflict with the above stated purpose of the club.

	\section{Article VII - Method of Dissolution of Organization}

	Should the club be forced to dissolve itself, any and all assets should be put toward the club's debt (if any). This includes university owned equipment or university funds.

	The remaining assets (hardware, operating funds, etc.) should be donated to a free software organization, such as the Free Software Foundation or the Electronic Frontier Foundation.  The organization to which the assets are donated must be determined at the time of dissolution ultimately by the Benevolent Dictator, with approval of the club advisor(s).\pagebreak

	\section{Additional By-Laws}
	\subsection{By-Law I - Method of Taking Attendance at Meetings}
	
	The steps listed below must be done in order, although it is not required to follow up one step immediately with the next:

		\begin{enumerate}
			\item At the start of each meeting,  \textbf{before} the class code associated with the club's account is needed to be displayed to members in attendance, a random alphanumeric code \textbf{must} be generated to be used as the new class code. Preferably a reputable random data generator service like Random.org is used to accomplish this task.  If no one present needs to refer to the class code, then one may skip to step 6.
			\item Configure properly the club account with the newly generated class code. If the said code conflicts with a pre-existing Remind.com class code, then repeat the code generation metioned in the previous step.
			\item Non-student members of the club will be asked to briefly step out of the meeting for the moment, if it is deemed that there is no better way to have the said members to not be able to have access to the new class code.
 			\item The code shall be displayed to the student members along with clear instructions on how to use the code to join the class associated with the club account. Note: unless under extenuating circumstances, copies of the code shall \textbf{not} be distributed to members by any means.  For members who've already joined the class, one may proceed to the next step.
 			\item The code and its associated instructions will now be completely removed from display, and any members asked to step outside in step 3 may be allowed back into the meeting.
 			\item A randomly generated English word (or the ``magic word")  will be displayed to all members in attendance, and an attendance announcement will be sent out to everyone in the class (associated with the club account) on Remind.com for each to reply to the announcement with the ``magic word".  There are no restrictions as to how the ``magic word" may be randomly generated.  It is highly recommended that this ``magic word" be kept on club records for future reference to verify records of attendance of members at meetings.
 			\item Based on replies to the announcement, members will be marked as present or absent accordingly using whatever means is most convenient, as long as the information can be kept on club records.\\
		\end{enumerate}

	\begin{center}
		\includegraphics [height=2.5em] {cc-0.png}
	\end{center}

\end{document}
